\documentclass[main]{subfiles}

\begin{document}

\section{Further Reading}
This script tries to summarize the material with complementary exercises and examples. The exercises are inspired by former exam questions, but also include some other questions which might occur. The examples are simply meant to re-enforce the reader's understanding of the topics. For those who would like to have a look at some additional learning materials, be it out of interest or because the script wasn't enough to foster a full understanding of the lecture's contents, could have a look at the following materials:
\begin{itemize}
    \item Lecture Slides/Recordings: The actual lecture materials should not be disregarded. This script only covers some of the material and is not enough to obtain a proper understanding of the field of Parallel Programming.
    \item Official Exercises: The exercises in this script only cover some of the discussed topics and its solutions were created by students like yourselves. It's strongly recommended that a student also goes through the official exercises in detail.
    \item  \textit{A Sophomoric Introduction to Shared-Memory Parallelism and Concurrency} by Dan Grossman: This document goes further into detail on and beyond some of the discussed topics. It served as a strong inspiration for this script's contents.
    \item \textit{The Art of Multiprocessor Programming} by Maurice Herlihy and Nir Shavit: A lot of the concepts presented during the second part of the course are based on this book. Those who are interested in the field of Distributed Computing or who would like to see more detailed explanations of topics such as Linearizability or Consensus will find this book to be very interesting.
    \item \textit{An Introduction to Parallel Algorithms} by Joseph Jájá: Some of the concepts and algorithms presented during the course were taken from this book. While it goes far beyond what is presented during the course, it might be interesting for extremely motivated students.
    \item \href{https://deadlockempire.github.io/}{Deadlockempire}: Those who find the concepts presented in this script difficult to wrap their heads around or who have difficulty recognizing race conditions or bad interleavings can build a good amount of intuition by playing this fun little game. It's short and could serve as a comfortable change of pace from regular studies.
\end{itemize}

\end{document}